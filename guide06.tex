\documentclass{uebblatt}

\begin{document}

\section*{Guide zu Übungsblatt 6}

Zu \textbf{Aufgabe 2}: Mit~$\M_c$ ist die volle Unterkategorie der kofasernden
Objekte gemeint. In der Vorlesung gab es ein ähnliches Lemma, an dessen Beweis
man sich orientieren kann; man benötigt lediglich noch ein weiteres Argument
vorab.

Zu \textbf{Aufgabe 3}: Die Kategorie~$\M/A$ hat als Objekte
\[ \text{Morphismen~$X \to A$ in~$\M$} \]
und als Morphismen
\[ \Hom_{\M/A}(X \xra{f} A,\, Y \xra{g} A) \defeq \{ h \in \Hom_\M(X,Y) \,|\,
  g \circ h = f \}. \]
Man muss nicht nachweisen, dass~$\M/A$ wieder eine Modellstruktur ist; das
haben wir (in einer spezielleren Situation) schon in Aufgabe~5c) von Blatt~4
gemacht. Ein Morphismus in~$\M/A$ ist per Definition genau dann eine schwache
Äquivalenz, eine Kofaserung oder eine Faserung, wenn er aufgefasst als
Morphismus in~$\M$ eine schwache Äquivalenz, eine Kofaserung bzw. eine Faserung
ist.

Der Begriff \emph{kofasernd erzeugt} fiel in der Vorlesung noch nicht. Er
bedeutet dasselbe wie \emph{kombinatorisch}, nur ohne die Bedingung, dass die
Kategorie lokal präsentierbar sein muss. Also ganz explizit: Eine
Modellkategorie~$\M$ heißt genau dann \emph{kofasernd erzeugt}, wenn
Mengen~$\I, \J \subseteq \Mor \M$ existieren, sodass~$\C = \Cof(\I)$ und~$\C \cap
\W = \Cof(\J)$. Dabei ist~$\Cof(\K)$ die Klasse der Retrakte von
relativen~$\K$-Zellkomplexen.

Der Vergissfunktor~$U : \M/A \to \M$ erhält und erzeugt Kolimiten. Das
bedeutet, dass sich ein Kolimes in~$\M/A$ einfach als Kolimes des
zugrundeliegenden Diagramms in~$\M$ berechnet (und man die Kolimesspitze auf
kanonische Art und Weise mit einem Morphismus nach~$A$ hinein ausstattet).
Insbesondere bewahrt~$U$ daher Pushouts und transfinite Kompositionen.

Für die Aufgabe nicht relevant, aber vielleicht dennoch interessant ist die
Tatsache, dass~$U$ im Allgemeinen nicht Limiten erhält oder erzeugt. Eine
Ausnahme bilden Pullbacks: Diese werden in~$\M/A$ genauso berechnet wie
in~$\M$. (Bei der Koscheibenkategorie~$A/\M$ ist es gerade umgekehrt.)

Für das Verständnis der Aufgabe ist die folgende Intuition nicht notwendig. Ich
finde sie trotzdem interessant. \emph{Ein Objekt aus~$\M/A$ stellt man sich
als~$A$-indizierte Familie von Objekten aus~$\M$ vor.} Dieser Spruch ergibt formal
keinen Sinn, denn~$A$ ist keine Menge. Die Sprache kommt vom Spezialfall~$\M =
\Set$: Eine Abbildung~$f : X \to A$ kann man sich nämlich tatsächlich
als~$A$-indizierte Familie von Mengen vorstellen, denn zu jedem~$a \in A$ gibt
es die Faser~$f^{-1}[\{a\}]$. Der Vergissfunktor~$\M/A \to \M$ berechnet dann
zu einer~$A$-indizierten Familie ihren Totalraum.

Wenn du Gefallen an Homotopietyptheorie hast, dann überlege doch (im Fall~$\M =
\Set$), ob der Vergissfunktor in eine Kette von Adjunktionen passt.
(Spoiler: Ja, und das hat mit der abhängigen Summe und dem abhängigen Produkt
zu tun.)

Zu \textbf{Aufgabe 4}: Details und Hintergründe zu~$\PSh(\C)$ stehen auch im
\href{http://pizzaseminar.speicherleck.de/skript1/pizzaseminar.pdf}{Skript zum ersten
Pizzaseminar} (ab Seite~42). Die Yoneda-Einbettung~$y$ schickt ein Objekt~$X
\in \C$ auf die Prägarbe~$\Hom_\C(\smallplaceholder, X)$. \emph{Prägarbe} ist
in diesem Zusammenhang nur ein Synonym für \emph{Funktor von~$\C^\op$
nach~$\Set$}. \emph{Lokal endlich-präsentierbar} bedeutet lokal
$\aleph_0$-präsentierbar.

Zu \textbf{Aufgabe 5}: Ist~$X$ irgendeine Menge, so zeigt das bekannte
Diagonalargument von Cantor, dass die Potenzmenge~$\P(X)$ größere Mächtigkeit
als~$X$ hat. Dadurch sieht man, dass es zu jeder Kardinalzahl eine noch größere
gibt. Eine Nachfolgerkardinalzahl ist nun eine solche, die direkt nach einer
anderen Kardinalzahl kommt. (Je zwei Kardinalzahlen sind gleich, größer oder
kleiner, daher ergibt diese Definition Sinn. Die Klasse der Kardinalzahlen ist
also total geordnet.)

Schreibe also~$\kappa = \lambda^+$ -- sei~$\kappa$ also die nächste
Kardinalzahl nach~$\lambda$. Zeige dann: Eine Vereinigung von~$\leq
\lambda$ vielen Mengen, die jeweils~$\leq \lambda$ viele Elemente enthalten,
enthält selbst nur~$\leq \lambda$ viele Elemente.

Verwende folgendes Lemma (ohne oder mit Beweis, je nach Stimmung): Ist~$\lambda$ eine unendlich große
Kardinalzahl, so gilt~$\lambda \cdot \lambda = \lambda$.

\end{document}
