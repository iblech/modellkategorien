\documentclass{uebblatt}

\begin{document}

\maketitle{10}{}

\begin{aufgabe}{Der klassifizierende Raum einer Gruppe}
\begin{enumerate}
\item Zeige, dass eine simpliziale Menge~$X$ genau dann ein Kan-Komplex ist,
wenn für alle Zahlen~$n \geq 0$ und~$k$ mit~$0 \leq k \leq n + 1$ folgende Bedingung erfüllt ist:
\begin{quote}Sind Simplizes~$x_0,\ldots,x_{k-1},x_{k+1},\ldots,x_{n+1} \in X_n$
mit~$d_i x_j = d_{j-1} x_i$ für alle~$i < j$ (wobei~$i$ und~$j$ ungleich~$k$)
vorgegeben, so existiert ein Simplex~$y \in X_{n+1}$ mit~$d_i(y) = x_i$ für
alle~$i \neq k$.\end{quote}
\item Zeige, dass jede simpliziale Gruppe ein Kan-Komplex ist.
\item Sei~$G$ eine Gruppe. Zeige, dass~$BG$ ein Kan-Komplex ist.
\end{enumerate}
\end{aufgabe}

\begin{aufgabe}{Skelett und Koskelett}
Die Kategorie~$\sSet_{\leq n}$ der~\emph{$n$-abgeschnittenen simplizialen Mengen}
ist die Kategorie der Funktoren~$\Delta_{\leq n}^\op \to \Set$,
wobei~$\Delta_{\leq n}$ die volle Unterkategorie von~$\Delta$ der
Objekte~$[0],\ldots,[n]$ ist.
\begin{enumerate}
\item Welchen kanonischen Funktor~$\sSet \to \sSet_{\leq n}$ gibt es?
\item Finde einen Linksadjungierten zu dem Funktor aus~a).
\item Finde einen Rechtsadjungierten zu dem Funktor aus~a).
\item Deute die Funktoren aus~b) und~c) geometrisch.
\end{enumerate}
\end{aufgabe}

\begin{aufgabe}{Wegzusammenhangskomponenten}
Sei~$X$ eine simpliziale Menge. Finde einen kanonischen Isomorphismus~$\pi_0(X)
\to \pi_0(|X|)$.
\end{aufgabe}

\begin{aufgabe}{Rückzug längs Monomorphismen}
Sei~$f : A \to B$ ein Monomorphismus in~$\sSet$. Sei~$X$ ein Kan-Komplex.
Zeige, dass der induzierte Morphismus~$X^B \to X^A$ eine Kan-Faserung ist.
\end{aufgabe}

\vfill
\centering
\href{http://www.anodynegame.com/}{\includegraphics[scale=0.85]{images/anodyne}}
\par

\end{document}
