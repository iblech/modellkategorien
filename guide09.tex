\documentclass{uebblatt}

\begin{document}

\section*{Guide zu Übungsblatt 9}

\section*{Erzeuger und Relationen in der Algebra}

Gruppen oder Moduln spezifiert man gelegentlich durch Angabe von
\emph{Erzeugern} und \emph{Relationen} zwischen diesen Erzeugern.

\begin{itemize}
\item Die additive Gruppe der ganzen Zahlen ist die von einem Erzeuger frei
erzeugte Gruppe:~$\ZZ = \langle x \rangle$.
\item Der additive Monoid der natürlichen Zahlen ist der von einem Erzeuger
frei erzeugte Monoid:~$\NN = \langle x \rangle$.
\item Die Gruppe~$\ZZ/(2)$ ist die von einem Erzeuger~$x$, modulo der
Relation~$x \circ x = e$ (neutrales Element), erzeugte Gruppe:~$\ZZ/(2) =
\langle x \,|\, x \circ x = e \rangle$.
\item Die Dieder-Gruppe~$D_n$ ist durch zwei Erzeuger und eine Relation
erzeugt:~$D_n = \langle r,s \,|\, r^n = e, srs = r^{-1} \rangle$.
\item Das Tensorprodukt~$M \otimes_A N$ kann wie folgt präsentiert werden:~$M
\otimes_A N = \langle (x,y) \,|\, x \in M, y \in N, (x,y_1+y_2) =
(x,y_1)+(x,y_2), \ldots \rangle$.
\end{itemize}

Ist ein algebraisches Objekt~$X$ durch Erzeuger und Relationen gegeben, so kann
man Morphismen in ein weiteres Objekt~$Y$ einfach dadurch spezifizieren, indem
man für jeden Erzeuger von~$X$ jeweils ein gewisses Element von~$Y$ als Bild
vorgibt und darauf achtet, dass diese Bilder die gegebenen Relationen erfüllen.

Außerdem kann man ein und dieselbe Präsentation durch Erzeuger und Relationen
in verschiedenen Kontexten interpretieren. Etwa ist~$\langle x,y \rangle$ in
der Kategorie der Gruppen die von zwei Elementen frei erzeugte Gruppe~$\ZZ
\star \ZZ$. Dieselbe Präsentation führt in der Kategorie der abelschen Gruppen
zur abelschen Gruppe~$\ZZ^2$.


\section*{Erzeugnisse als Kolimiten}

In jeder Kategorie algebraischer Objekte kann man die Erzeugnisse von
Präsentationen als gewisse Kolimiten charakterisieren. Sei~$\ul{n}$ das von~$n$
Erzeugern~$e_1,\ldots,e_n$ ohne Relationen erzeugte Objekt: in~$\Grp$ also~$\ZZ
\star \cdots \star \ZZ$, in~$\Ab$ ist es~$\ZZ^n$, in~$\mathrm{Mod}(R)$ ist
es~$R^n$. In~$\Set$ is es irgendeine~$n$-elementige Menge.

\begin{itemize}
\item $\ZZ \star \ZZ$ ist in der Kategorie der Gruppen der Kolimes des
folgenden Diagramms:
\[ \ul{1} \qquad \ul{1} \]
\item $\langle x,y,z \,|\, 2x = 3y \rangle$ ist in der Kategorie der abelschen
Gruppen der Kolimes des folgenden Diagramms:
\[ \xymatrix{
  \ul{1} \ar[rd]_{e_1 \mapsto e_1} && \ul{1} \ar[ld]^{e_1 \mapsto e_2} & \ul{1} \\
  & \ul{2} \\
  & \ul{1} \ar@/^/[u]^{e_1 \mapsto 2e_1} \ar@/_/[u]_{e_1 \mapsto 3e_2}
} \]
Kleinere Diagramme sind ebenfalls möglich (etwa~$\ul{1} \to \ul{2} \quad
\ul{1}$), aber weniger systematisch.
\end{itemize}

Die allgemeine Regel lautet also wie folgt:
\begin{itemize}
\item Für jeden Erzeuger~$x$ platziert man eine Kopie von~$\ul{1}$ im
Diagramm:~$\ul{1}_x$.
\item Für jede Relation~$r$ der Form~$t_1(a_1,\ldots,a_n) =
t_2(a_1,\ldots,a_n)$ platziert man eine Kopie von~$\ul{n}$,
notiert~$\ul{n}_r$; eine Kopie von~$\ul{1}$, notiert~$\ul{1}^r$; und
folgende Morphismen:

Für~$i=1,\ldots,n$ einen Morphismus~$\ul{1}_{a_i} \to \ul{n}_r$, der
den Erzeuger von~$\ul{1}_{a_i}$ auf~$e_i$ schickt; einen
Morphismus~$\ul{1}^r \to \ul{n}_r$, der den Erzeuger von~$\ul{1}^r$
auf~$t_1(e_1,\ldots,e_n)$ schickt; einen Morphismus~$\ul{1}^r \to \ul{n}_r$,
der den Erzeuger von~$\ul{1}^r$ auf~$t_2(e_1,\ldots,e_n)$.
\end{itemize}

\end{document}
