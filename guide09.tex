\documentclass{uebblatt}

\begin{document}

\section*{Guide zu Übungsblatt 9}

Zu \textbf{Aufgabe 2} folgt weiter unten eine ausführliche Erklärung. Die
Aufgabe soll euch damit vertraut machen, wie man simpliziale Mengen durch
Erzeuger und Relationen angeben kann. Das ist wichtig, denn wenn man Bilder von
simplizialen Mengen malt, macht man das implizit! Schließlich zeichnet niemand
die Unmengen an degenerierten Simplizes dazu.

Zu \textbf{Aufgabe 3}: Das \emph{$n$-Skelett}~$\sk_n X$ einer simplizialen
Menge~$X$ ist die simpliziale (Unter-)Menge mit
\[ (\sk_n X)_p \defeq \{ x \in X_p \,|\,
  \exists q \leq n,\, f : [p] \to [q],\, y \in X_q{:}\ 
  x = X(f)y \} \subseteq X_p. \]
Die Wirkung auf monotone Abbildungen~$g$, also~$(\sk_n X)(g)$, definiert man als
Einschränkung der Abbildung~$X(g)$.

Anschaulich entsteht das~$n$-Skelett einer simplizialen Menge indem man nur die
Simplizes in Graden~$\leq n$ behält. Im wörtlichen Sinn ist das nicht ganz
wahr, denn das~$n$-Skelett enthält trotzdem auch Simplizes beliebig hoher
Dimension; diese sind aber Degenerationen von Simplizes der Dimension~$\leq n$.

Die simpliziale Menge~$\SS^{n-1}$ diejenige simpliziale Untermenge
von~$\Delta[n]$, der das eindeutig bestimmte nichtdegenerierte~$n$-Simplex
fehlt:
\[ \SS^{n-1}_m \defeq \{ f : [m] \to [n] \,|\,
  \text{$f$ ist monoton und nicht surjektiv} \} \subseteq \Delta[n]_m. \]
Anschaulich ist~$\SS^{n-1}$ eine simpliziale Version der~$(n-1)$-Sphäre.

Das \emph{Koprodukt} (disjunkt-gemachte Vereinigung) von simplizialen Mengen
wird einfach levelweise berechnet. Daher gilt
\[ \Bigl(\coprod_{x \in I} \SS^{n-1}\Bigr)_m =
  \coprod_{x \in I} \SS^{n-1}_m =
  \{ (x, f) \,|\, x \in I,\ f \in \SS^{n-1}_m \}. \]

Zu \textbf{Teilaufgabe 3a)}: Grüßt Yoneda von mir, wenn ihr diese Teilaufgabe
bearbeitet!

Zu \textbf{Teilaufgabe 3c)}: Mit~$X_{(n)} \subseteq X_n$ bezeichnen wir die
Menge der nichtdenerierten~$n$-Simplizes. Dabei heißt ein~$n$-Simplex~$x
\in X_n$ genau dann \emph{nichtdegeneriert}, wenn es eine \emph{keine} surjektive monotone
Abbildung~$f : [n] \to [m]$ derart gibt, dass~$m < n$ und dass~$x = X(f)(y)$ für ein~$y
\in X_m$. Wenn dir das hilft, dann verwende folgende grundlegende Beobachtungen
(die du keinesfalls beweisen musst):
\begin{enumerate}
\item Jede monotone Abbildung lässt sich faktorisieren in einer surjektive
monotone Abbildung gefolgt von einer injektiven monotonen Abbildung.
\item Ein~$n$-Simplex~$x \in X_n$ ist genau dann nichtdegeneriert, wenn aus~$x
= X(f)y$ für eine monotone Abbildung~$f : [n] \to [m]$ und~$y \in X_m$ folgt,
dass~$f$ injektiv ist.
\item Für jedes~$n$-Simplex~$x \in X_n$ gibt es ein eindeutiges Paar~$(f,y)$
bestehend aus einem nichtdegenerierten Simplex~$y \in X_m$ und einer
surjektiven monotonen Abbildung~$f : [n] \to [m]$ mit~$x = X(f)y$.
\end{enumerate}

Das Pushout-Diagramm drückt aus, dass sich das~$n$-Skelett einer simplizialen
Menge aus ihrem~$(n-1)$-Skelett durch Einkleben der
nichtdegenerierten~$n$-Simplizes ergibt.

Um die universelle Eigenschaft nachzuweisen, kann folgender Tipp hilfreich
sein. Ein Simplex~$x \in (\sk_n X)_m$ lässt sich auf eindeutige Art und
Weise in der Form~$x = X(f)u$ schreiben, wobei~$f : [m] \to [\ell]$ eine
Surjektion,~$u \in X_\ell$ nichtdegeneriert und~$\ell \leq n$ ist. Es genügt
schon, die gesuchte Abbildung auf den nichtdegenerierten Simplizes von~$\sk_n
X$ zu definieren -- die restlichen Werte sind durch das Axiom an simpliziale
Abbildungen schon festgelegt (inwiefern?).

Zu \textbf{Aufgabe 4}: Ein \emph{innerer Kan-Komplex} ist eine simpliziale
Menge, in der sich alle \emph{inneren} Hörner füllen lassen. Jeder Kan-Komplex
ist auch ein innerer Kan-Komplex, aber nicht umgekehrt. Ihr könnt euch über
diese Aufgabe freuen, liefert sie doch ein Beispiel -- vielleicht das erste,
das ihr seht -- für eine~$(\infty,1)$-Kategorie!

Wenn du Schwierigkeiten hast, dir herzuleiten, wie der Morphismenanteil
von~$N\C$ aussieht, dann googel oder schreib mir eine Mail!

Die zu einem topologischen Raum~$Y$ gehörige simpliziale Menge~$\Sing Y$ hat
als~$n$-Simplizes die stetigen Abbildungen~$\Delta^n \to Y$. Dabei
ist~$\Delta^n \subseteq \RR^{n+1}$ das topologische Standard-$n$-Simplex.


\section*{Erzeuger und Relationen in der Algebra}

Gruppen oder Moduln spezifiert man gelegentlich durch Angabe von
\emph{Erzeugern} und \emph{Relationen} zwischen diesen Erzeugern.

\begin{itemize}
\item Die additive Gruppe der ganzen Zahlen ist die von einem Erzeuger frei
erzeugte Gruppe:~$\ZZ = \langle x \rangle$.
\item Der additive Monoid der natürlichen Zahlen ist der von einem Erzeuger
frei erzeugte Monoid:~$\NN = \langle x \rangle$.
\item Die Gruppe~$\ZZ/(2)$ ist die von einem Erzeuger~$x$, modulo der
Relation~$x \circ x = e$ (neutrales Element), erzeugte Gruppe:~$\ZZ/(2) =
\langle x \,|\, x \circ x = e \rangle$.
\item Die Dieder-Gruppe~$D_n$ ist durch zwei Erzeuger und eine Relation
erzeugt:~$D_n = \langle r,s \,|\, r^n = e, srs = r^{-1} \rangle$.
\item Das Tensorprodukt~$M \otimes_A N$ kann wie folgt präsentiert werden:~$M
\otimes_A N = \langle (x,y) \,|\, x \in M, y \in N, (x,y_1+y_2) =
(x,y_1)+(x,y_2), \ldots \rangle$.
\end{itemize}

Ist ein algebraisches Objekt~$X$ durch Erzeuger und Relationen gegeben, so kann
man Morphismen in ein weiteres Objekt~$Y$ einfach dadurch spezifizieren, indem
man für jeden Erzeuger von~$X$ jeweils ein gewisses Element von~$Y$ als Bild
vorgibt und darauf achtet, dass diese Bilder die gegebenen Relationen erfüllen.

Außerdem kann man ein und dieselbe Präsentation durch Erzeuger und Relationen
in verschiedenen Kontexten interpretieren. Etwa ist~$\langle x,y \rangle$ in
der Kategorie der Gruppen die von zwei Elementen frei erzeugte Gruppe~$\ZZ
\star \ZZ$. Dieselbe Präsentation führt in der Kategorie der abelschen Gruppen
zur abelschen Gruppe~$\ZZ^2$.


\section*{Erzeugnisse als Kolimiten}

In jeder Kategorie algebraischer Objekte kann man die Erzeugnisse von
Präsentationen als gewisse Kolimiten charakterisieren. Sei~$\ul{n}$ das von~$n$
Erzeugern~$e_1,\ldots,e_n$ ohne Relationen erzeugte Objekt: in~$\Grp$ also~$\ZZ
\star \cdots \star \ZZ$, in~$\Ab$ ist es~$\ZZ^n$, in~$\mathrm{Mod}(R)$ ist
es~$R^n$. In~$\Set$ is es irgendeine~$n$-elementige Menge.

\begin{itemize}
\item $\ZZ \star \ZZ$ ist in der Kategorie der Gruppen der Kolimes des
folgenden Diagramms:
\[ \ul{1} \qquad \ul{1} \]
\item $\langle x,y,z \,|\, 2x = 3y \rangle$ ist in der Kategorie der abelschen
Gruppen der Kolimes des folgenden Diagramms:
\[ \xymatrix{
  \ul{1} \ar[rd]_{e_1 \mapsto e_1} && \ul{1} \ar[ld]^{e_1 \mapsto e_2} & \ul{1} \\
  & \ul{2} \\
  & \ul{1} \ar@/^/[u]^{e_1 \mapsto 2e_1} \ar@/_/[u]_{e_1 \mapsto 3e_2}
} \]
Kleinere Diagramme sind ebenfalls möglich (etwa~$\ul{1} \to \ul{2} \quad
\ul{1}$), aber weniger systematisch.
\end{itemize}

Die allgemeine Regel lautet also wie folgt:
\begin{itemize}
\item Für jeden Erzeuger~$x$ platziert man eine Kopie von~$\ul{1}$ im
Diagramm:~$\ul{1}_x$.
\item Für jede Relation~$r$ der Form~$t_1(a_1,\ldots,a_n) =
t_2(a_1,\ldots,a_n)$ platziert man eine Kopie von~$\ul{n}$,
notiert~$\ul{n}_r$; eine Kopie von~$\ul{1}$, notiert~$\ul{1}^r$; und
folgende Morphismen:

Für~$i=1,\ldots,n$ einen Morphismus~$\ul{1}_{a_i} \to \ul{n}_r$, der
den Erzeuger von~$\ul{1}_{a_i}$ auf~$e_i$ schickt; einen
Morphismus~$\ul{1}^r \to \ul{n}_r$, der den Erzeuger von~$\ul{1}^r$
auf~$t_1(e_1,\ldots,e_n)$ schickt; einen Morphismus~$\ul{1}^r \to \ul{n}_r$,
der den Erzeuger von~$\ul{1}^r$ auf~$t_2(e_1,\ldots,e_n)$.
\end{itemize}


\section*{Erzeuger und Relationen für simpliziale Mengen}

Genauso wie man Gruppen oder Moduln durch Erzeuger und Relationen angeben kann,
kann man auch simpliziale Mengen durch Erzeuger und Relationen konstruieren.
Da es simplizialen Mengen aber an Operationen im herkömmlichen Sinn wie
"`Summe"' oder "`Produkt"' mangelt, ist diese Vorstellung anfangs vielleicht
etwas ungewohnt.

Lasst uns zunächst klären, wie man gewöhnliche Mengen (statt simpliziale
Mengen) durch Erzeuger und Relationen beschreiben kann. Eine Menge, die von
drei Erzeugern und ohne Relationen erzeugt wird, ist zum Beispiel~$\{ a,b,c
\}$. Relationen sind nicht so spannend: Man kann nur fordern, dass manche der
gegebenen Erzeuger gleich sein sollen. Zum Beispiel ist die Menge, die von drei
Erzeugern~$a$,~$b$ und~$c$ modulo der Relation~$b = c$ erzeugt wird, die
Menge~$\{ a,b,c \}/{\sim}$, wobei die Äquivalenzrelation~$b$ mit~$c$
identifiziert und sonst nichts. Das Erzeugt ist also isomorph zu~$\{ D,E \}$.

Nun zu simplizialen Mengen. Erzeuger sind Simplizes in vorgegebenen
Dimensionen. Relationen können mit den Korand- und Koentartungsabbildungen
verschiedene Simplizes miteinander in Relation setzen. Zum Beispiel kann man
fordern: Die fünfte Ecke von dem und dem~8-Simplex soll gleich der dritten Ecke
von dem und dem anderen~6-Simplex sein.

In Aufgabe~2 vom Übungsblatt sollt ihr raten, was die Erzeugnisse in den drei
Fällen sind, diese möglichst explizit angeben (was sind die Simplizes in jeder
Dimension?) und analog wie beim Abschnitt \emph{Erzeugnisse als Kolimiten} das
Diagramm hinmalen, dessen Kolimes die erzeugte simpliziale Menge ist.

Ein Erzeuger in Grad~$n$ muss dabei durch eine Kopie von~$\Delta[n]$ im
Diagramm repräsentiert werden. Die "`allgemeine Regel"' oben könnt ihr auch
ignorieren, wenn ihr auf andere Art und Weise das richtige Diagramm findet.

\end{document}
