\documentclass{uebblatt}

\begin{document}

\section*{Guide zu Übungsblatt 12}

Bei \textbf{Aufgabe 3} kommen \emph{punktierte Mengen} vor. Eine punktierte
Menge ist nichts anderes als eine gewöhnliche Menge, bei der ein Element als
"`Basispunkt"' besonders ausgezeichnet ist. Die Homotopiegruppen~$\pi_n(X,x)$ werden
konventionsgemäß durch die Festlegung der konstanten
Abbildung~$\Delta[n]/\partial \Delta[n] \to X$ mit Wert~$x$ als Basispunkt zu
punktierten Mengen. Eine Sequenz~$A \xra{f} B \xra{g} C$ punktierter Mengen heißt genau
dann exakt, wenn~$\operatorname{im} f = \operatorname{ker} g$; dabei
ist~$\operatorname{ker} g \defeq \{ b \in B \,|\, g(b) = (\text{Basispunkt
von~$C$}) \}$.

Außerdem kommt bei Aufgabe~3 die \emph{Faser} von~$p$ über~$x$ vor. Diese ist
über das folgende Pullbackdiagramm definiert:
\[ \xymatrix{
  F \ar[r]\ar[d] & E \ar[d] \\
  1 \ar[r] & X
} \]
Dabei ist~$1$ die terminale simpliziale Menge, also~$\Delta[0]$, und~$1 \to X$
ist die Abbildung, die den einzigen Punkt von~$1$ auf~$x$ schickt.

Bei Aufgabe~5 ist ein ähnliches Pullbackdiagramm relevant, um~$p^{-1}[A]$ zu
definieren:
\[ \xymatrix{
  p^{-1}[A] \ar[r]\ar[d] & E \ar[d] \\
  A \ar[r] & X
} \]

\end{document}
