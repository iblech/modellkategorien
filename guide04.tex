\documentclass{uebblatt}
\begin{document}

\section*{Guide zu Übungsblatt 4}

Hier eine Präzisierung bzw. ein Tipp zu \textbf{Aufgabe 1}. In der Rückrichtung
hat man eine Rechts-Kan-Erweiterung~$(F,\,\varepsilon : F G \Rightarrow \Id_\M)$
gegeben und muss daraus eine Adjunktion~$F \dashv G$ basteln. Das dafür
nötige~$\varepsilon$ hat man ja bereits gegeben. Wie konstruiert man~$\eta$?
Wie weist man die Dreiecksidentitäten nach?

Bei der Hinrichtung hat man eine Adjunktion~$(F \dashv G, \eta, \varepsilon)$
gegeben und muss aus~$F$ eine Rechts-Kan-Erweiterung von~$\Id_\M$ längs~$G$
machen. Außerdem muss man zeigen, dass~$G$ diese Erweiterung bewahrt. Man kann
sogar zeigen, dass \emph{jeder} Funktor~$H$, der bei~$\M$ startet, diese
Erweiterung bewahrt -- das ist genauso schwer und spart sogar noch
Schreibaufwand, da man im Spezialfall~$H = \Id_\M$ das Resultat mitbeweist,
dass~$F$ selbst eine Rechts-Kan-Erweiterung ist.

Zur Erinnerung: Man sagt genau dann, dass ein Funktor~$H : \A \to \B$ eine
Rechts-Kan-Erweiterung~$(R,\,\varepsilon : RK \Rightarrow T)$ von~$T : \M \to
\A$ längs~$K : \M \to \C$ erhält, wenn~$(HR,\,H\varepsilon : HRK \Rightarrow
T)$ eine Rechts-Kan-Erweiterung von~$HT$ längs~$K$ ist.

\end{document}
