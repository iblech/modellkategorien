\documentclass{uebblatt}
\begin{document}

\section*{Guide zu Übungsblatt 4}

Hier eine Präzisierung bzw. ein Tipp zu \textbf{Aufgabe 1}. In der Rückrichtung
hat man eine Rechts-Kan-Erweiterung~$(F,\,\varepsilon : F G \Rightarrow \Id_\M)$
gegeben und muss daraus eine Adjunktion~$F \dashv G$ basteln. Das dafür
nötige~$\varepsilon$ hat man ja bereits gegeben. Wie konstruiert man~$\eta$?
Wie weist man die Dreiecksidentitäten nach?

Bei der Hinrichtung hat man eine Adjunktion~$(F \dashv G, \eta, \varepsilon)$
gegeben und muss aus~$F$ eine Rechts-Kan-Erweiterung von~$\Id_\M$ längs~$G$
machen. Außerdem muss man zeigen, dass~$G$ diese Erweiterung bewahrt. Man kann
sogar zeigen, dass \emph{jeder} Funktor~$H$, der bei~$\M$ startet, diese
Erweiterung bewahrt -- das ist genauso schwer und spart sogar noch
Schreibaufwand, da man im Spezialfall~$H = \Id_\M$ das Resultat mitbeweist,
dass~$F$ selbst eine Rechts-Kan-Erweiterung ist.

Zur Erinnerung: Man sagt genau dann, dass ein Funktor~$H : \A \to \B$ eine
Rechts-Kan-Erweiterung~$(R,\,\varepsilon : RK \Rightarrow T)$ von~$T : \M \to
\A$ längs~$K : \M \to \C$ erhält, wenn~$(HR,\,H\varepsilon : HRK \Rightarrow
T)$ eine Rechts-Kan-Erweiterung von~$HT$ längs~$K$ ist.

Ein Tipp für \textbf{Aufgabe 3}, der auch an sich eine schöne Übung im Umgang
mit Diagrammen abgibt, ist: Sind das linke und rechte Quadrat jeweils
Pushout-Diagramme, so ist auch das Gesamtrechteck ein Pushout-Diagramm.
\[ \xymatrix{
  A \ar[r]\ar[d] & B \ar[r]\ar[d] & C \ar[d] \\
  D \ar[r] & E \ar[r] & F
} \]

Zu \textbf{Aufgabe 5c)}: Die Objekte der Kategorie~$\M_\star$ sind Morphismen der
Form~$1 \to X$ in~$\M$ (mit~$X \in \M$ beliebig). Dabei bezeichnet~$1 \in \M$
das terminale Objekt in~$\M$. Die Morphismen sind kommutative Dreiecke.
Bei dieser Aufgabe ist einiges zu tun. Zunächst mal muss man zeigen,
dass~$M_\star$ wieder alle kleinen Limiten und Kolimiten enthält. Holt euch
dazu Tipps ab! Anschließend muss man geeignet schwache Äquivalenzen, Faserungen
und Kofaserungen definieren. Holt euch auch dafür Tipps ab!

Bei \textbf{Aufgabe 6} gibt es einen Tipp, der die Aufgabe deutlich vereinfacht.
Bitte holt ihn euch bei mir ab. Die Notation soll jedenfalls andeuten, dass der
gegebene Morphismus~$A \to B$ eine Kofaserung und dass~$X \to Y$ eine
azyklische Faserung ist. Hängt euch gegebenenfalls nicht daran auf, bei dem von
euch konstruierten Zylinder die Kofaserungseigenschaft nachzuweisen.

\end{document}

Das Tensorprodukt von Funktoren in \textbf{Aufgabe 2} ist wie folgt definiert.
Sei~$G : \C^\op \to \Set$ und~$F : \C \to \Set$. Dann ist~$G \otimes F$ ein
Objekt von~$\Set$, und zwar das Koende
\[ G \otimes F \defeq \int^c G(c) \times F(c). \]
