\documentclass{uebblatt}

\newcommand{\stopper}{\begin{center}$\heartsuit$\end{center}}
\begin{document}

\section*{Guide zu Übungsblatt 5}

Für \textbf{Aufgabe~2} kann man folgendes filtrierte (sogar gerichtete) System
von Räumen betrachten.\footnote{Das Beispiel stammt von Marc Olschok in einer
\href{https://golem.ph.utexas.edu/category/2009/05/journal_club_geometric_infinit_3.html\#c023818}{Diskussion
auf dem n-Kategorien-Café}.}
Fixiere eine nicht-offene Teilmenge~$A \subseteq X$. Definiere für~$n \in \NN$ die Menge~$Y_n \defeq X \amalg
\NN$ und versehe diese Menge mit folgender Topologie: Die einzigen
nichttrivialen offenen Mengen sollen von der Form~$A \amalg \{ m \in \NN \,|\,
m \geq k \}$ sein (mit~$k \geq n$ beliebig). Wähle als
Übergangsabbildungen~$Y_n \to Y_{n+1}$ die (mengentheoretischen) Identitäten.
Dann ist der Kolimes der indiskrete Raum auf der Menge~$X
\amalg \NN$ (das bedeutet, dass die einzigen offenen Mengen die leere Menge und
der gesamte Raum sind) und die kanonische Injektion~$X \to X \amalg \NN$ faktorisiert
nicht über einen der Räume~$Y_n$.

\textbf{Hinweis zu Aufgabe~3}. Aufgrund der vielen Indizes, Kegel und Kokegel
wird die Lösung möglicherweise unübersichtlich. Wer mag, kann folgende
abgespeckte Aufgabe bearbeiten. Sie enthält genau dieselben interessanten
Aspekte, kommt aber mit weniger Technik aus.

Sei~$C$ eine Partialordnung (zum Beispiel~$C = \NN$) und~$D$ eine Menge.
Sei für jedes Paar~$(c,d) \in C \times D$ eine Teilmenge~$F_{cd}$ einer
gewissen Obermenge~$\Omega$ gegeben.
\begin{enumerate}
\item Zeige: Ganz ohne weitere Voraussetzungen gilt
\[ \bigcup_{c \in C} \bigcap_{d \in D} F_{cd} \subseteq \bigcap_{d \in D} \bigcup_{c \in C} F_{cd}.
\]
\item Sei nun~$C$ sogar gerichtet\footnote{Das bedeutet, dass~$C$ mindestens
ein Element enthält und je zwei Elemente~$x,y \in C$ eine gemeinsame obere
Schranke besitzen: ein~$z \in C$ mit~$x \preceq z$ und~$y \preceq z$.} und~$D$
endlich. Dann gilt auch~"`$\supseteq$"'.
\item Wer dann noch die Puste hat, kann die Aufgabe ins Unendliche eskalieren:
Sei~$\kappa$ eine reguläre Kardinalzahl. Sei~$D$ zwar nicht endlich, aber
gelte~$|D| < \kappa$. Sei~$C$ $\kappa$-gerichtet -- das heißt, dass je~$<
\kappa$ viele Elemente eine gemeinsame obere Schranke besitzen. Dann gilt
ebenfalls~"`$\supseteq$"'.
\end{enumerate}

Noch eine Bemerkung zur ursprünglichen Aufgabe: Es ist tatsächlich wichtig,
dass der Funktor~$F$ in die Kategorie der Mengen (und nicht sonst irgendeine
Kategorie) abbildet.

\textbf{Tipp zu Aufgabe~4}. Die Rückrichtung ist etwas einfacher als die
Hinrichtung und auch schon sehr interessant (schließlich konstruieren wir ein
Pseudoinverses; es ist doch spannend, woher das kommt!). Verwende folgendes
Theorem:
\begin{quote}Sei~$X$ kofasernd. Sei~$p : Z \stackrel{\sim}{\twoheadrightarrow} Y$
eine azyklische Faserung. Dann ist~$p_\star : \pi^\ell(X,Z) \to
\pi^\ell(X,Y),\,[f] \mapsto [p \circ f]$ eine Bijektion.
\end{quote}
Für geeignete Wahl von~$X$,~$Z$,~$Y$ und~$p$ ist nämlich die Identität ein
Element der rechten Seite und besitzt dann ein Urbild. Eine Faktorisierung des
gegebenen Morphismus könnte hilfreich sein.

Bei \textbf{Aufgabe~5a)} kann man die Abbildung in die umgekehrte Richtung
explizit angeben. Denkt dran, dass fasernder und kofasernder Ersatz mit
kanonischen Morphismen daher kommen. Es ist auch nützlich, sich die Diagramme
aufzumalen, die für einen Morphismus~$f : X \to Y$ den Morphismus~$RQf$
definieren.

Zur technischen Vereinfachung der Aufgabe kann man ausnutzen, dass ohne
Beschränkung der Allgemeinheit~$RQX = RX$ und~$RQY = QY$ gilt; schließlich ist~$X$
schon kofasernd und~$Y$ schon fasernd.

Tipps zu \textbf{Aufgabe 5b)} können bei mir per Mail abgeholt werden!

\stopper

\textbf{Endliche Mengen in konstruktiver Mathematik.} Dieser Abschnitt ist
(nur) für die Fans konstruktiver Mathematik unter euch gedacht.
Eine Menge~$X$ heißt genau dann \emph{Kuratowski-endlich}, wenn es eine
eine Surjektion~$[n] \twoheadrightarrow X$ gibt (mit~$n \geq 0$), wobei~$[n] =
\{0,1,\ldots,n-1\}$. Eine Menge~$X$ heißt genau dann \emph{endlich}, wenn es
eine Bijektion~$[n] \stackrel{\cong}{\to} X$ gibt.

Von Teilmengen von (Kuratowski-)endlichen Mengen kann man im Allgemeinen nicht
zeigen, dass sie wieder (Kuratowski-)endlich sind. Es sind aber Quotienten von
Kuratowski-endlichen Mengen wieder Kuratowski-endlich. Endliche Mengen sind
stets diskret, Kura\-towski-endliche Mengen im Allgemeinen aber nicht.
Konstruktiv gibt es noch weitere Endlichkeitsbegriffe, und man kann alle auch
so formulieren, dass sie nicht Bezug nehmen auf die (unendlich große) Menge der
natürlichen Zahlen.

\begin{aufgabe*}{Bonusaufgabe 1}{Kompakte Objekte in der Kategorie der Mengen konstruktiv}

Zeige: In der Kategorie der Mengen ist eine Menge~$X$ genau dann
Kuratowski-endlich, wenn~$\Hom_\Set(X,\smallplaceholder)$ mit filtrierten
Kolimiten von Monomorphismen vertauscht; und genau dann endlich,
wenn~$\Hom_\Set(X,\smallplaceholder)$ mit beliebigen filtrierten Kolimiten
vertauscht.

\emph{Bemerkung:} Konstruktiv heißt eine Kategorie genau dann filtriert, wenn
alle Diagramme mit endlicher (nicht Kuratowski-endlicher) Indexkategorie einen
Kokegel zulassen. Das ist äquivalent dazu, dass die Kategorie ein Objekt
enthält, zu je zwei Objekten ein Möchtegernkoprodukt enthält und zu je zwei
parallelen Morphismen ein Möch\-te\-gern\-ko\-dif\-fe\-renz\-kern enthält.
\end{aufgabe*}
\vspace{-1em}

\stopper

\textbf{Angst vor bösen Formulierungen.}
Mich persönlich hat es immer etwas beunruhigt, dass die meisten AutorInnen die
1-kategorielle Definition der Lokalisierung verwenden, habe ich doch auf dem
nLab gelernt, dass diese in diesem Kontext im technischen Sinn \emph{böse} ist,
da sie Funktoren auf Gleichheit testet. Hinzu kommt noch, dass manche AutorInnen
in der 1-kategoriellen Definition zusätzlich fordern, dass der
Lokalisierungsfunktor~$Q : \C \to \C[\W^{-1}]$ auf Objekten bijektiv ist; das
ist doch eine unnatürliche Bedingung, die in universellen Eigenschaften nichts
verloren hat.

Sogar in dem wunderschönen Buch
\href{http://www.math.harvard.edu/~eriehl/cathtpy.pdf}{\emph{Categorical
Homotopy Theory}} von Emily
Riehl (frei online verfügbar), in dem der Begriff \emph{Kan-Erweiterung} mehr als 130 Mal vorkommt,
ist das so!

Wer diese Unstimmigkeiten auch beunruhigend findet, kann die folgende
Bonusaufgabe bearbeiten, um diese aus der Welt zu schaffen. Gern geschehen!

\begin{aufgabe*}{Bonusaufgabe 2}{Die 1- und 2-kategorielle Definition der Lokalisierung}
Sei~$\C$ eine Kategorie und~$\W$ eine Klasse von Morphismen in~$\C$. Dann ist
die \emph{1-kategorielle Lokalisierung} eine Kategorie~$\C[\W^{-1}]$ zusammen
mit einem Funktor~$Q : \C \to \C[\W^{-1}]$, welcher die Morphismen aus~$\W$ auf
Isomorphismen schickt, sodass folgende 1-kategorielle universelle Eigenschaft
erfüllt ist:
\begin{quote}Ist~$F : \C \to \D$ irgendein Funktor, der die Morphismen aus~$\W$
auf Isomorphismen schickt, so gibt es genau einen Funktor~$\bar F : \C[\W^{-1}]
\to \D$ mit~$\bar F \circ Q = F$. Anders formuliert: Für jede Kategorie~$\D$
induziert~$Q^\star$ eine Bijektion zwischen den Funktoren~$\C[\W^{-1}] \to \D$
und denjenigen Funktoren~$\C \to \D$, welche die Morphismen aus~$\W$ auf
Isomorphismen schicken.
\end{quote}
Wie in der Vorlesung diskutiert sollte man diese Eigenschaft allerdings
eigentlich nie fordern. Moralisch besser ist folgende 2-kategorielle
universelle Eigenschaft:
\begin{quote}
Für jede Kategorie~$\D$ induziert~$Q^\star$ eine Äquivalenz zwischen der
Kategorie der Funktoren~$\C[\W^{-1}] \to \D$ und der Kategorie derjenigen
Funktoren~$\C \to \D$, welche die Morphismen aus~$\W$ auf Isomorphismen
schicken.
\end{quote}
\begin{enumerate}
\item Zeige, dass eine Lösung des 1-kategoriellen Optimierungsproblems
eindeutig ist bis auf eindeutige Isomorphie.
\item Zeige, dass eine Lösung des 2-kategoriellen Optimierungsproblems
eindeutig ist bis auf Äquivalenz, wobei je zwei Äquivalenzen eindeutig isomorph
zueinander sind.
\item Zeige: Sei~$Q : \C \to \C[\W^{-1}]$ Lösung des 1-kategoriellen Problems.
Zeige, dass~$Q$ auf Objekten bijektiv ist, dass also die von~$Q$ induzierte
Abbildung~$\Ob \C \to \Ob \C[\W^{-1}]$ bijektiv ist.

\emph{Tipp:} Für Injektivität verwende in der universellen Eigenschaft für~$\D$
die Kategorie mit denselben Objekten wie~$\C$ und genau einem Morphismus
zwischen je zwei Objekten.
\item Zeige: Ist~$Q : \C \to \C[\W^{-1}]$ eine Lösung des 1-kategoriellen
Problems, so auch des 2-kategoriellen.

\emph{Tipp:} Verwende das Ergebnis aus Teilaufgabe~c).
\end{enumerate}
\end{aufgabe*}

\end{document}
