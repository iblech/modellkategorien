\documentclass{uebblatt}

\usepackage{draftwatermark}
\definecolor{pink}{rgb}{0.95,0.9,0.95}
\SetWatermarkText{\textsf{\textcolor{pink}{ENTWURF}}}
\SetWatermarkScale{1}

\begin{document}

\maketitle{7}{}

\begin{aufgabe}{Dichte Unterkategorien}
Ein Funktor~$K : \M \to \C$ heißt genau dann \emph{dicht}, wenn jedes Objekt~$c
\in \C$ Kolimes von Objekten der Form~$K(m)$ ist -- und zwar nicht irgendwie,
sondern auf die kanonische Art und Weise~$c = \colim_{K(m) \to c} K(m)$.
\begin{enumerate}
\item Zeige: Die Inklusion~$\{\{\heartsuit\}\} \to \Set$ der vollen
Unterkategorie von~$\Set$, welche nur das Objekt~$\{\heartsuit\}$ enthält, ist
dicht.
\item Zeige: Die Inklusion~$\{\RR\} \to \mathrm{Vect}(\RR)$ der vollen
Unterkategorie von~$\mathrm{Vect}(\RR)$, welche nur das Objekt~$\RR$ enthält, ist
nicht dicht.
\item Zeige: Für jede Kategorie~$\C$ ist die Yoneda-Einbettung~$y : \C \to
\PSh(\C)$ dicht.
\item Zeige, dass ein Funktor~$K : \M \to \C$ genau dann dicht ist,
wenn~$(\Id_\C,\,\id_K)$ eine punktweise Links-Kan-Erweiterung von~$K$
längs~$K$ ist.
\end{enumerate}
\end{aufgabe}

\begin{aufgabe}{Dichtheit der kompakten Objekte}
Sei~$S$ eine Menge $\kappa$-kompakter Objekte in einer Kategorie~$\C$,
sodass jedes Objekt aus~$\C$ $\kappa$-filtrierter Kolimes von Objekten aus~$S$
ist.
\begin{enumerate}
\item Zeige: Ein Objekt aus~$\C$ ist genau dann~$\kappa$-kompakt, wenn es
Retrakt von einem Objekt aus~$S$ ist.
\item Sei~$\alpha$ eine reguläre Kardinalzahl mit~$|S| < \alpha$
und~$|\Hom_\C(X,X)| < \alpha$ für alle~$X \in S$. Zeige, dass es~$< \alpha$
viele Isomorphieklassen von~$\kappa$-kompakten Objekten in~$\C$
gibt.
\item Zeige, dass die volle Unterkategorie der~$\kappa$-kompakten Objekte in~$\C$ dicht
in~$\C$ liegt.
\end{enumerate}
%\alpha reguläre Kardinalzahl, C lokal-präsentierbare Kategorie und G in C
%\alpha-kleine Menge von \alpha-kleinen Objekten, so daß jedes Objekt in C
%Kolimes über G ist. (Hierbei ist eine \alpha-kleine Menge eine mit
%Kardinalität < \alpha.)
%
%1. Zeige: G ist dicht in C. (Dicht ist noch nicht in der Vorlesung definiert worden.)
%2. Zeige: Sei H die Menge (?) der alpha-kleinen Kolimiten über G. Dann ist H \alpha-klein.
%3. Zeige: Jedes Objekt aus C ist \alpha-filtrierter Kolimes über H.
\end{aufgabe}

\begin{aufgabe}{Beispiele für nicht (endlich-)präsentierbare Kategorien}
\begin{enumerate}
\item Zeige, dass die Kategorie der topologischen Räume nicht lokal präsentierbar
ist.
\item Zeige, dass in der Kategorie der Banachräume und linearen Kontraktionen
das Objekt~$\RR$ nicht~$\aleph_0$-kompakt, aber~$\aleph_1$-kompakt ist.
\end{enumerate}
\end{aufgabe}

\begin{aufgabe}{Quillen-Adjunktionen auf Niveau der Homotopiekategorien}
Sei~$F \dashv U$ eine Quillen-Adjunktion. Zeige, dass
die folgenden Aussagen äquivalent sind:
\begin{itemize}
\item[1.] $F \dashv U$ ist eine Quillen-Äquivalenz.
\item[2.] $\LL F \dashv \RR U$ ist eine Äquivalenz der Homotopiekategorien.
\item[3.] $F$ reflektiert schwache Äquivalenzen zwischen kofasernden Objekten und
für alle fasernden Objekte~$Y$ ist~$FQUY \to FUY \xra{\varepsilon} Y$ eine schwache
Äquivalenz.
\item[4.] $U$ reflektiert schwache Äquivalenzen zwischen fasernden Objekten und
für alle kofasernden Objekte~$X$ ist~$X \xra{\eta} UFX \to URFX$ eine schwache
Äquivalenz.
\end{itemize}
\end{aufgabe}
\enlargethispage{1.2cm}

\centering
\includegraphics[height=2.8cm]{images/mmh-adjunctions-cookie-monster}
\par

\end{document}

Aufgaben zu abgeleiteten Funktoren
