\documentclass{uebblatt}

\usepackage{draftwatermark}
\definecolor{pink}{rgb}{0.95,0.9,0.95}
\SetWatermarkText{\textsf{\textcolor{pink}{ENTWURF}}}
\SetWatermarkScale{1}

\begin{document}

\maketitle{7}{}

\begin{aufgabe}{XXX}
%\alpha reguläre Kardinalzahl, C lokal-präsentierbare Kategorie und G in C
%\alpha-kleine Menge von \alpha-kleinen Objekten, so daß jedes Objekt in C
%Kolimes über G ist. (Hierbei ist eine \alpha-kleine Menge eine mit
%Kardinalität < \alpha.)
%
%1. Zeige: G ist dicht in C. (Dicht ist noch nicht in der Vorlesung definiert worden.)
%2. Zeige: Sei H die Menge (?) der alpha-kleinen Kolimiten über G. Dann ist H \alpha-klein.
%3. Zeige: Jedes Objekt aus C ist \alpha-filtrierter Kolimes über H.
\end{aufgabe}

\begin{aufgabe}{Nichtpräsentierbarkeit der Kategorie der topologischen Räume}
Zeige, dass die Kategorie der topologischen Räume nicht lokal präsentierbar
ist.
\end{aufgabe}

\begin{aufgabe}{Quillen-Adjunktionen auf Niveau der Homotopiekategorien}
Prop. 16.2.3 aus May/Ponto.
\end{aufgabe}

\end{document}

Aufgaben zu abgeleiteten Funktoren
evtl. separat: Dichtheit und Kan-Erweiterungen, kanonischer Kolimes in lok. präs. Kategorien
