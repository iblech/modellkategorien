\documentclass{uebblatt}

\begin{document}

\maketitle{12}{}

\begin{aufgabe}{Schwache Äquivalenzen zwischen simplizialen Mengen}
Sei~$f : X \to Y$ ein Morphismus simplizialer Mengen. Seien~$X \to
\overline{X}$ und~$Y \to \overline{Y}$ anodyne Erweiterungen,
wobei~$\overline{X}$ und~$\overline{Y}$ Kan-Komplexe sind. Zeige, dass~$f$
genau dann eine schwache Äquivalenz ist, wenn ein Lift~$\overline{f} :
\overline{X} \to \overline{Y}$ von~$f$ eine Homotopieäquivalenz ist.
\end{aufgabe}

\begin{aufgabe}{Modelle für den Einheitskreis}
Zeige oder widerlege: Die simpliziale Menge~$\SS^1$ (eine Untermenge
von~$\Delta[2]$) und der Quotient~$\Delta[1]/{\sim}$ aus Blatt~9, Aufgabe~2c)
sind zueinander homotopieäquivalent.
\end{aufgabe}

\begin{aufgabe}{Die lange exakte Sequenz von Homotopiegruppen}
Für einen Kan-Komplex~$X$ und einen Basispunkt~$x \in X_0$ ist die~\emph{$n$-te
Homotopiegruppe}~$\pi_n(X,x)$ die Menge der Homotopieklassen von
Basispunkt-bewahrenden Morphismen~$\Delta[n]/\partial \Delta[n] \to X$.
\begin{enumerate}
\item Sei~$p : E \to X$ eine Kan-Faserung,~$F \defeq p^{-1}[x]$ die Faser
über~$x$ und~$e \in F_0$ ein Punkt. Verwende die Rechtshochhebungseigenschaft
von~$p$ bezüglich der Inklusion~$\Lambda^0[n] \to \Delta[n]$, um eine
Abbildung~$\delta : \pi_n(X,x) \to \pi_{n-1}(F,e)$ zu konstruieren.
\item Zeige, dass folgende Sequenz punktierter Mengen exakt ist.
\[ \cdots \to \pi_n(F,e) \to \pi_n(E,e) \to \pi_n(X,x) \xra{\partial}
\pi_{n-1}(F,e) \to \cdots \to \pi_0(F) \to \pi_0(E) \to \pi_0(X) \]
\end{enumerate}
\end{aufgabe}
\vspace{-1em}

\begin{aufgabe}{Das Theorem von Whitehead für Kan-Komplexe}
Sei~$f : X \to Y$ ein Morphismus von Kan-Komplexen. Seien die induzierten
Abbildungen~$\pi_0(X) \to \pi_0(Y)$ und~$\pi_n(X,x) \to \pi_n(Y,f(x))$ für
alle~$x \in X_0$ und~$n \geq 1$ bijektiv. Zeige, dass~$f$ eine
Homotopieäquivalenz ist.
\end{aufgabe}

\begin{aufgabe}{Rückzüge von starken Deformationsretrakten}
Sei~$p : E \to X$ eine Serre-Faserung und~$i : A \to X$ ein Monomorphismus.
Sei~$A$ in~$X$ ein starker Deformationsretrakt. Zeige, dass~$p^{-1}[A]$ in~$E$
ein starker Deformationsretrakt ist.
\end{aufgabe}

\end{document}
