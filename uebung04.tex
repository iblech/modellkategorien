\documentclass{uebblatt}
\usepackage{stmaryrd}

\begin{document}

\maketitle{4}{}

\begin{aufgabe}{Adjunktionen als Kan-Erweiterungen}
Sei~$G : \M \to \C$ ein Funktor (oder ein Morphismus in einer beliebigen
2-Kategorie). Zeige, dass~$G$ genau dann einen Linksadjungierten besitzt, wenn
eine Rechts-Kan-Erweiterung von~$\Id_\M$ längs~$G$ existiert und~$G$ diese
bewahrt.
\end{aufgabe}

\begin{aufgabe}{Retrakte in Modulkategorien}
Sei~$i : U \hookrightarrow M$ die Inklusion eines Untermoduls. Zeige: Genau
dann ist~$U$ ein direkter Summand von~$M$, wenn~$i$ ein Linksinverses besitzt.
\end{aufgabe}

\begin{aufgabe}{Zellkomplexe und Koprodukte}
Sei~$\I$ eine Menge von Morphismen in einer kovollständigen Kategorie.
Sei~$X$ ein Objekt. Sei~$f : A \to B$ ein relativer~$\I$-Zellkomplex. Zeige: Der
induzierte Morphismus~$A \amalg X \to B \amalg X$ ist wieder ein
relativer~$\I$-Zellkomplex.
\end{aufgabe}

\begin{aufgabe}{Zylinder und Überlagerungen}
Sei $E \xra{\pi} X$ eine Überlagerung. Sei~$\pt \xra{\iota} [0,1]$ die
Inklusion eines Intervallendes. Zeige~$\iota \boxslash \pi$.
\end{aufgabe}
\vspace{-1em}

\begin{aufgabe}{Erste Schritte mit Modellstrukturen}
\begin{enumerate}
\item Zeige, dass die Kategorie der Mengen zusammen mit den Isomorphismen,
Injektionen und Surjektionen eine Modellkategorie bildet.
\item Zeige, dass in einer Modellkategorie je zwei der folgenden fünf Klassen
(außer $(\C,\F \cap \W)$ und $(\C \cap \W,\F)$) die anderen eindeutig
festlegen: $\W$, $\C$, $\F$, $\C \cap \W$, $\F \cap \W$.
\item Sei~$\M$ eine Modellkategorie. Baue auf der Kategorie~$\M_\star$ der
punktierten Objekte in~$\M$ eine Modellstruktur.
\end{enumerate}
\end{aufgabe}

\begin{aufgabe}{Eindeutigkeit von Lifts}
Sei in einer Modellkategorie ein Quadrat mit zwei Lifts~$h$ und~$k$ gegeben. \\
Zeige, dass~$h$ und~$k$ zueinander sehr gut linkshomotop sind.
\marginpar{\vspace*{-2.0cm}\hspace*{-2.6cm}$
  \xymatrixcolsep{3pc}\xymatrixrowsep{3pc}
  \xymatrix{
    A \ar[r] \ar@{^{(}->}[d] & X \ar@{->>}[d]^{\sim} \\
    B \ar@/^/@{-->}[ru]^h \ar@/_/@{-->}[ru]_k \ar[r] & Y
  }
$}
\end{aufgabe}

\vfill
\centering
\href{http://fashions-cloud.com/pages/c/cat-lifting-weights/}{\includegraphics[height=3.5cm]{images/lifting-property-1}}
\qquad
\href{http://www.woophotos.com/weight-lifter/}{\includegraphics[height=3.5cm]{images/lifting-property-2}}
\par

\end{document}

\begin{aufgabe}{Hom als Delta-Distribution}
Sei~$F : \C^\op \to \Set$ eine Prägarbe und~$c \in \C$. Zeige:
$\Hom_\C(\smallplaceholder, c) \otimes F = F(c)$.
\end{aufgabe}

Zeige, dass die transfinite Komposition von relativen~$\I$-Zellkomplexen
ein relativer~$\I$-Zellkomplex ist.
