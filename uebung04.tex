\documentclass{uebblatt}

\usepackage{draftwatermark}
\definecolor{pink}{rgb}{0.95,0.9,0.95}
\SetWatermarkText{\textsf{\textcolor{pink}{ENTWURF}}}
\SetWatermarkScale{1}

\begin{document}

\maketitle{4}{}

\begin{aufgabe}{Adjunktionen als Kan-Erweiterungen}
Sei~$G : \M \to \C$ ein Funktor (oder ein Morphismus in einer beliebigen
2-Kategorie). Zeige, dass~$G$ genau dann einen Linksadjungierten besitzt, wenn
eine Rechts-Kan-Erweiterung von~$\Id_\M$ längs~$G$ existiert und~$G$ diese
bewahrt.
\end{aufgabe}

\begin{aufgabe}{Hom als Delta-Distribution}
Sei~$F : \C^\op \to \Set$ eine Prägarbe und~$c \in \C$. Zeige:
$\Hom_\C(\smallplaceholder, c) \otimes F = F(c)$.
\end{aufgabe}

\begin{aufgabe}{Retrakte in Modulkategorien}
Sei~$i : U \hookrightarrow M$ die Inklusion eines Untermoduls. Zeige: Genau
dann ist~$U$ ein direkter Summand von~$M$, wenn~$i$ ein Linksinverses besitzt.
\end{aufgabe}

\begin{aufgabe}{Komposition von Pushout-Quadraten}
Zeige: Sind das linke und rechte Quadrat jeweils Pushout-Diagramme, so ist auch
das Gesamtrechteck ein Pushout-Diagramm.
\[ \xymatrix{
  A \ar[r]\ar[d] & B \ar[r]\ar[d] & C \ar[d] \\
  D \ar[r] & E \ar[r] & F
} \]
\end{aufgabe}

Fehlt noch: Beweise aus der Vorlesung \ldots

\vfill
\centering
\href{http://fashions-cloud.com/pages/c/cat-lifting-weights/}{\includegraphics[height=4cm]{images/lifting-property-1}}
\qquad
\href{http://www.woophotos.com/weight-lifter/}{\includegraphics[height=4cm]{images/lifting-property-2}}
\par

\end{document}

Interessante Retrakte von Morphismen in Top?
