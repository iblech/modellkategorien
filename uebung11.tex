\documentclass{uebblatt}

\usepackage{draftwatermark}                                                                          
\definecolor{pink}{rgb}{0.95,0.9,0.95}                                                               
\SetWatermarkText{\textsf{\textcolor{pink}{ENTWURF}}}                                                
\SetWatermarkScale{1}                                                                                

\begin{document}

\maketitle{11}{}

\begin{aufgabe}{Ein erster Einblick in Homotopiepushouts}
\begin{enumerate}
\item Zeige anhand eines Beispiels in~$\Top$, dass folgende wünschenswerte
Aussage im Allgemeinen falsch ist: Sei~$X \leftarrow A \rightarrow Y$ ein
Diagramm. Sei~$X' \leftarrow A' \rightarrow Y'$ ein dazu schwach äquivalentes
Diagramm. Dann ist auch der Pushout~$X \amalg_A Y$ schwach äquivalent zu~$X'
\amalg_{A'} Y'$.
\end{enumerate}
Der \emph{Homotopiepushout} eines Diagramms~$X \leftarrow A \rightarrow
Y$ in einer Modellkategorie ist per Definition der Pushout eines schwach
äquivalenten Diagramms~$X' \leftarrow A' \rightarrow Y'$, in dem~$A'$ kofasernd
und die beiden Morphismen Kofaserungen sind.
\begin{enumerate}
\addtocounter{enumi}{1}
\item Zeige, dass der Homotopiepushout bis auf schwache Äquivalenz
wohldefiniert ist.
\item Sei die Modellkategorie linkseigentlich und sei einer der Morphismen~$X
\leftarrow A$ und~$A \rightarrow Y$ eine Kofaserung. Zeige, dass dann der
Homotopiepushout und der gewöhnliche Pushout übereinstimmen.
\end{enumerate}
\end{aufgabe}

\begin{aufgabe}{Anodynizität von Kofaserungen}
Zeige: Eine Kofaserung zwischen simplizialen Mengen (bezüglich der
Quil\-len-Mo\-dell\-struk\-tur) ist genau dann anodyn, wenn sie eine schwache
Äquivalenz ist.
\end{aufgabe}

\begin{aufgabe}{Fundamentalgruppe der eindimensionalen Sphäre}
Berechne~$\pi_1(\SS^1)$.
\end{aufgabe}

\begin{aufgabe}{Erzeugnis unter beliebigen vs. filtrierten Kolimiten}
Sei~$S$ eine Menge~$\kappa$-kompakter Objekte in einer kovollständigen
Kategorie~$\C$. Sei jedes Objekt aus~$\C$ kleiner Kolimes von Objekten aus~$S$.
Zeige, dass jedes Objekt aus~$\C$ dann ein~\emph{$\kappa$-fil\-trierter} Kolimes
von Objekten aus~$\overline{S}$ ist, wobei~$\overline{S}$ der Abschluss von~$S$
unter~$\kappa$-kleinen Kolimiten ist.
\end{aufgabe}

\end{document}

* http://ncatlab.org/nlab/show/subdivision
* Abschnitt 1.8 aus Joyal/Tierney
* Kovervollständigung und so weiter
