\documentclass{uebblatt}

\begin{document}

\maketitle{11}{}

\begin{aufgabe}{Ein erster Einblick in Homotopiepushouts}
\begin{enumerate}
\item Zeige anhand eines Beispiels in~$\Top$, dass folgende wünschenswerte
Aussage im Allgemeinen falsch ist: Sei~$X \leftarrow A \rightarrow Y$ ein
Diagramm. Sei~$X' \leftarrow A' \rightarrow Y'$ ein dazu schwach äquivalentes
Diagramm. Dann ist auch der Pushout~$X \amalg_A Y$ schwach äquivalent zu~$X'
\amalg_{A'} Y'$.
\end{enumerate}
Der \emph{Homotopiepushout} eines Diagramms~$X \leftarrow A \rightarrow
Y$ in einer Modellkategorie ist per Definition der Pushout eines schwach
äquivalenten Diagramms~$X' \leftarrow A' \rightarrow Y'$, in dem~$A'$ kofasernd
und die beiden Morphismen Kofaserungen sind.
\begin{enumerate}
\addtocounter{enumi}{1}
\item Zeige, dass der Homotopiepushout bis auf schwache Äquivalenz
wohldefiniert ist.
\item Sei die Modellkategorie linkseigentlich und sei einer der Morphismen~$X
\leftarrow A$ und~$A \rightarrow Y$ eine Kofaserung. Zeige, dass dann der
Homotopiepushout und der gewöhnliche Pushout übereinstimmen.
\end{enumerate}
\end{aufgabe}

\end{document}
