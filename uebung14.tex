\documentclass{uebblatt}

\usepackage{draftwatermark}
\definecolor{pink}{rgb}{0.95,0.9,0.95}
\SetWatermarkText{\textsf{\textcolor{pink}{ENTWURF}}}
\SetWatermarkScale{1}

\newcommand{\Ch}{\mathrm{Ch}}

\begin{document}

\maketitle{14}{}

\end{document}

Aus dem Hovey:

1) Prop. 5.1.4
2) Cor. 5.1.5.
3) Lemma 5.2.6 (mit Tip)
4) Zylinderobjekte in Ch_*(R)
5) Kosimpliziale Auflösungen in Ch_*(R) und Berechnung von A \otimes^L K
für A \in Ch_*(R), K \in sSet (Tip: Dold-Kan)

Die Operation von S = Ho sSet auf Ho Ch_*(R) nochmals für S_* = Ho sSet_*
punktiert und mit dem Smash-Produkt. Was ist dann A \smash S^1?

Berechnen Sie die kanonische Faser- und Kofasersequenz für die Kategorie
der Kettenkomplexe (Siehe 6.1 und 6.2 in Hovey.)

Gruppen und Ko-Gruppenstruktur auf \Omega A und \Sigma A
