\documentclass{uebblatt}
\begin{document}

\section*{Guide zu Übungsblatt 3}

Für \textbf{Aufgaben~2 und~4} benötigt man \emph{Links}-Kan-Erweiterungen. Die genaue
Definition ist folgende. Eine Links-Kan-Erweiterung von~$T : M \to A$ längs~$K
: M \to C$ ist ein Tupel~$(L,\eta)$ bestehend aus
\begin{itemize}
\item einem Morphismus~$L : C \to A$ und
\item einem 2-Morphismus~$\eta : T \Rightarrow L \circ K$
\end{itemize}
sodass gilt:
Für jedes Tupel~$(L' : C \to A,\eta' : T \Rightarrow L' \circ K)$ existiert
genau ein 2-Morphismus~$\sigma : L \Rightarrow L'$ mit~$
\sigma K \circ \eta = \eta'$.

Für \textbf{Aufgabe~2} ist folgende Charakterisierung nützlich: Ein Funktor~$L : C \to
A$ ist genau dann Links-Kan-Erweiterung von~$T$ längs~$K$, wenn es in~$S \in
[C,A]$ natürliche Isomorphismen
\[ \Hom_{[C,A]}(L, S) \cong \Hom_{[M,A]}(T, S \circ K) \]
gibt. Mit dieser Aussage und dem (klassischen) Yoneda-Lemma kann man
die Aufgabe mit viel Freude lösen. Ihr werdet nicht mehr als eine Zeile
benötigen.

Für \textbf{Aufgabe~4} benötigt man die \emph{Kolimesformel} für
Links-Kan-Erweiterungen; dank der Volltreuheit von~$K$ vereinfacht sich in der
Aufgabe diese ganz enorm. Die Formel lautet:
\[ (\mathrm{Lan}_K(T))(c) = \colim_{f:K(m) \to c} T(m). \]
Die genaue Aussage dazu ist folgende (dual zu Aufgabe~3): Wenn für jedes
Objekt~$c \in \C$ der genannte Kolimes existiert, kann man diese Zuordnung zu
einem Funktor~$\mathrm{Lan}_K(T) : \C \to \A$ ausdehnen, und dieser Funktor
wird dann zusammen mit der Transformation~$\eta : T \Rightarrow \mathrm{Lan}_K(T)
\circ K$ (welche als Komponenten~$\eta_m : T(m) \to \mathrm{Lan}_K(T)(K(m))$
kanonische Morphismen in den Kokegel hat) zu einer Links-Kan-Erweiterung
von~$T$ längs~$K$.

In \textbf{Aufgabe~5} ist mit~$\mathrm{succ}$ die Abbildung~$\NN \to \NN$, $x \mapsto x
+ 1$ gemeint. Teil~c) muss natürlich nur bearbeiten, wer Garben kennt (und Fan
von ihnen ist).

In der \textbf{Bonusaufgabe} meint~$D^2$ den vollen Einheitskreis, also~$\{ (x,y) \in
\RR^2 \,|\, x^2 + y^2 \leq 1 \}$. Die Einbettung~$S^1 \hookrightarrow D^2$ soll
die Einheitskreislinie auf die Randlinie von~$D^2$ abbilden.

\end{document}
