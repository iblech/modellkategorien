\documentclass{uebblatt}

\begin{document}

\section*{Guide zu Übungsblatt 10}

Für \textbf{Aufgabe 1} muss man in Teilaufgabe~b) wissen, was eine simpliziale
Gruppe ist. Dazu gibt es drei äquivalente Definitionen; am praktischsten ist
vermutlich die erste:
\begin{enumerate}
\item Eine simpliziale Gruppe ist wie eine simpliziale Menge, nur dass man auf
den Simplexmengen~$X_n$ noch jeweils eine Gruppenstruktur hat und dass die
Abbildungen~$X(f) : X_m \to X_n$ für~$f : [n] \to [m]$ alle
Gruppenhomomorphismen sind.
\item Eine simpliziale Gruppe ist ein Funktor~$\Delta^\op \to \Set$ zusammen
mit der Angabe einer Faktorisierung über den Vergissfunktor~$\Grp \to \Set$.
\item Eine simpliziale Gruppe ist ein Gruppenobjekt in der Kategorie der
simplizialen Mengen.
\end{enumerate}

Außerdem kommt in Teilaufgabe~c) die simpliziale Menge~$BG$ vor. Kurz und knapp
definiert ist diese nichts anderes als der Nerv (siehe Blatt~9) der von~$G$
induzierten Kategorie (diese hat genau ein Objekt, nämlich~$\star$, und für
jedes Gruppenelement einen Morphismus~$\star \to \star$). Ganz explizit gilt
also~$(BG)_n = G^n$. Achtung, Falle: Die simpliziale Menge~$BG$ kann genau dann
mit einer Gruppenstruktur versehen werden, wenn~$G$ abelsch ist. Teilaufgabe~c)
folgt also nicht aus~b).

Wenn ihr Indexschlachten vermeiden möchtet, beweist in den Teilaufgaben~b)
und~c) nur die Fälle niedriger Dimension.

\textbf{Aufgabe 2} hat etwas mit Kan-Erweiterungen zu tun. Wenn du den
Zusammenhang herausgefunden hast, kannst du die Limes- bzw. Kolimesformel
verwenden, um die gesuchten Funktoren zumindest formal hinzuschreiben; dann
sollte die Darstellung nur noch vereinfacht werden. Auch, wenn man keine
Formeln findet, ist Teilaufgabe~d) machbar.

Wer Fan von Enden und Koenden ist, kann auch folgende Formeln verwenden:
\[ \mathrm{Lan}_K(T)(c) = \int^m \Hom_\C(Km,c) \cdot Tm \qquad
  \mathrm{Ran}_K(T)(c) = \int_m (Tm)^{\Hom_\C(c,Km)}. \]
Wer dagegen Kan-Erweiterungen nicht mag, kann die Aufgabe natürlich auch
ohne sie bearbeiten. Ihr könnt es auch dabei belassen, Links- und
Rechtsadjungierte zum Funktor~$\sSet_{\leq N+1} \to \sSet_{\leq n}$ zu finden;
das ist technisch einfacher und enthält auch schon den eigentlichen Inhalt
dieser Aufgabe.

In \textbf{Aufgabe 3} ist~$|X|$ die geometrische Realisierung von~$X$
und~$\pi_0(|X|)$ die Menge ihrer Wegzusammenhangskomponenten, also die Menge
der Punkte von~$|X|$ modulo der Äquivalenzrelation, die genau dann zwei Punkte
miteinander identifiziert, wenn sie durch einen stetigen Weg verbunden werden
können.

In \textbf{Aufgabe 4} ist der Morphismus~$f^\star : X^B \to X^A$ der eindeutig
bestimmte Morphismus mit der Eigenschaft, dass das Diagramm
\[ \xymatrix{
  X^B \times B \ar[r] \ar[d]_{f^\star \times \id_B} & X \ar[d]^f \\
  Y^B \times B \ar[r] & Y
} \]
kommutiert. Als Tipp kann ich anbieten: Universelle Eigenschaft des
Funktionenkomplex und der Satz von Gabriel--Zisman.

\end{document}
