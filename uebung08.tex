\documentclass{uebblatt}

\begin{document}

\maketitle{8}{}

\begin{aufgabe}{Was tun, wenn das terminale Objekt nicht kofasernd ist?}
\emph{Diese Aufgabe ist offen gestellt und soll zum Experimentieren einladen.}
Bekanntlich gibt es folgende Aussage (Lemma~16.4.9 in May/Ponto): Sei~$\V$ eine
kartesisch abgeschlossene monoidale Modellkategorie. Sei das terminale
Objekt~$\star$ kofasernd. Dann wird die Modellkategorie~$\V_\star$ der punktierten
Objekte mit dem Smash-Produkt zu einer monoidalen Modellkategorie.

Sei in diesem Kontext~$\star$ nicht kofasernd. Welche Bedingung könnte man
stellen, damit~$QS^0 \wedge X \to S^0 \wedge X$ immer noch für alle kofasernden
Objekte~$X$ eine schwache Äquivalenz ist? Dabei ist~$S^0$ das Einsobjekt
von~$\V_\star$.
\end{aufgabe}

\end{document}

* Thema: Monoidale Kategorien und Moduln und Algebren über diesen.
