\documentclass{uebblatt}

\begin{document}

\section*{Guide zu Übungsblatt 7}

Die Indexkategorie im Kolimes bei der Definition von Dichtheit in
\textbf{Aufgabe 1} sieht wie folgt aus. Die Objekte dieser Kategorie sind
\[ \text{Morphismen der Form~$K(m) \to c$ in~$\C$ mit~$m \in \M$ beliebig.} \]
Die Morphismen sind
\[ \Hom(K(m) \xra{f} c,\, K(n) \xra{g} c) \defeq
  \{ h : m \to n \,|\, g \circ K(h) = f \}. \]

Die vier Teilaufgaben von Aufgabe~1 sind voneinander unabhängig. Ich hoffe,
dass mindestens eine gefällt! Für \textbf{Teilaufgabe~b) von Aufgabe~1}
überlege dir gut, wie das fragliche Diagramm in diesem Fall aussieht. Skizze!
Welche Morphismen verlaufen in diesem Diagramm? Für \textbf{Teilaufgabe~c) von
Aufgabe~1} ist eigentlich nur eine bestimmte Übungsaufgabe von Blatt~6 zu
zitieren; welche? Das Yoneda-Lemma lautet (in seiner konventionellen Form):
\begin{quote}
Sei~$\C$ eine Kategorie. Sei~$A$ ein Objekt von~$\C$. Sei~$F : \C^\op \to \Set$
ein Funktor. Dann ist die kanonische Abbildung
\[ \Hom_{[\C^\op,\Set]}(y(A), F) \stackrel{\cong}{\longrightarrow} F(A), \]
welche eine natürliche Transformation~$\eta$ auf~$\eta_A(\id_A)$ schickt, eine
Bijektion. Außerdem ist diese Abbildung natürlich in~$A$ und~$F$.
\end{quote}

Verwende für \textbf{Teilaufgabe~d) von Aufgabe~1} unbedingt die Kolimesformel
für die punktweise Links-Kan-Erweiterung; mit der ist die Lösung in wenigen
Zeilen machbar, ohne sie wird die Aufgabe deutlich komplizierter. Die Formel
lautet:
\[ (\mathrm{Lan}_K(T))(c) = \colim_{f:K(m) \to c} T(m). \]

Verwende für die Hinrichtung in \textbf{Teilaufgabe~a) von Aufgabe~2} einfach
die Definition von~$\kappa$-Kompaktheit und die Voraussetzung. Verwende für die
Rückrichtung folgende zwei Aussagen: $\kappa$-kleine Kolimiten (insbesondere
endliche Kolimiten) von $\kappa$-kompakten Objekten sind~$\kappa$-kompakt, und
Retrakte können stets als Kolimiten geschrieben werden (wenn~$f \circ g = \id :
U \to U$, dann ist~$U$ Kolimes der Diagrams~$X \rightrightarrows X$, wobei der
eine Morphismus die Identität und der andere~$g \circ f$ ist).

Hole dir für \textbf{Teilaufgabe~b) von Aufgabe~2} bei mir einen Tipp ab.
Minimaltipp vorab: Retrakte haben etwas mit idempotenten Morphismen zu tun;
sie werden durch solche klassifiziert. Der Morphismus~$g \circ f$ aus dem
vorherigen Absatz ist ein solcher idempotenter Morphismus.

Tipp für \textbf{Teilaufgabe~c) von Aufgabe~2}: Es ist zu zeigen, dass der
Inklusionsfunktor von der vollen Unterkategorie der $\kappa$-kompakten Objekte
in die Kategorie~$\C$ dicht ist. Ist~$X \in \C$ ein beliebiges Objekt, so gibt
es nach Voraussetzung ein Diagramm (d.\,h. einen Funktor)~$F : \I \to \C$
mit~$\kappa$-filtrierter Quellkategorie~$I$ und $F(i) \in S$ für alle~$i \in \I$
sodass~$X = \colim F$. Zum Vergleich haben wir das kanonische Diagramm~$G :
(\C_{\kappa c}/X) \to \C$, wobei~$\C_{\kappa c}/X$ die Kategorie der
$\kappa$-kompakten Objekte über~$X$ bezeichnet. Nach Definition von Dichtheit
ist zu zeigen, dass~$X = \colim G$. Versuche dazu, eine Beziehung zwischen~$\I$
und~$\C_{\kappa c}/X$ herzuleiten! Findest du einen Funktor~$\I \to \C_{\kappa
c}/X$? Wenn ja, bist du fast fertig; mit Proposition~5.39 ("`Kolimiten
kofinaler Unterdiagramme"') aus dem
\href{http://pizzaseminar.speicherleck.de/skript1/pizzaseminar.pdf}{Skript zum
ersten Pizzaseminar} folgt sofort die Behauptung.

Verwende bei \textbf{Aufgabe 3a)} ohne Beweis, dass jeder nicht-diskrete Raum für keine
Kardinalzahl~$\kappa$ $\kappa$-kompakt ist. In Aufgabe~2 von Blatt~5 hatten wir
gesehen, dass nicht-diskrete Räume zumindest nicht $\aleph_0$-kompakt sind; der
Beweis dort überträgt sich auf die allgemeinere Behauptung.

Ich persönlich finde \textbf{Teilaufgabe b) von Aufgabe 3} spannender.
Wer den Grad an benötigter Technik reduzieren möchte, kann folgende Behauptung
beweisen.
\begin{quote}
Sei~$I$ eine Partialordnung.
Sei~$(V_i)_{i \in I}$ eine monotone Familie von abgeschlossenen Unterräumen
eines Banachraums~$V$ (\emph{monoton} heißt: $i \preceq i' \Rightarrow
V_i \subseteq V_{i'}$).
Sei ein Vektor aus dem topologischen Abschluss der Vereinigung der~$V_i$
gegeben:~$x \in \overline{\bigcup_i V_i}$.
\begin{itemize}
\item Setze von~$I$ nur voraus, dass~$I$ gerichtet ist.
Zeige anhand eines expliziten Gegenbeispiels, dass dann im Allgemeinen nicht
folgt, dass~$x \in V_j$ für ein~$j \in J$.
\item Setze nun an~$I$ voraus, dass~$I$ $\aleph_1$-gerichtet ist. (Das
bedeutet, dass jede Teilmenge~$J \subseteq I$ mit~$|J| < \aleph_1$ eine obere
Schranke in~$I$ besitzt.) Zeige, dass dann durchaus folgt, dass~$x \in V_j$ für
ein~$j \in J$.
\end{itemize}
\end{quote}
Die Zahl~$\aleph_1$ ist die nach~$\aleph_0$ nächstgrößere Kardinalzahl. Analog
wie für eine Menge~$X$ genau dann~$|X| < \aleph_0$ gilt, wenn~$X$ endlich ist,
gilt genau dann~$|X| < \aleph_1$, wenn~$X$ (endlich oder) abzählbar ist.

Wer die eigentliche Aufgabe bearbeite möchte, sollte sich zunächst überlegen,
wie filtrierte Kolimiten in der Kategorie der Banachräume und linearen
Kontraktionen berechnet werden. (Wie ist die Norm zu definieren? Ist die eigene
Vermutung über den Kolimes wieder vollständig?) Außerdem sollte man bedenken,
dass~$\Hom(\RR,\smallplaceholder)$ in dieser Kategorie \emph{nicht} die
zugrundeliegende Menge, sondern den zugrundeliegenden (abgeschlossenen)
Einheitsball berechnet. Eine lineare Abbildung~$A$ zwischen
Banachräumen heißt genau dann \emph{Kontraktion}, wenn~$\|Ax\| \leq \|x\|$ für
alle~$x$ aus dem Quellraum.

Übrigens: Die Kategorie der Banachräume ist zwar nicht lokal
$\aleph_0$-präsentierbar, wohl aber lokal $\aleph_1$-präsentierbar. Das hat
sogar eine erstaunlich einfache Begründung (siehe Übung).

Zu \textbf{Aufgabe~4} ist zu bemerken, dass per Definition ein Funktor~$F$
genau dann schwache Äquivalenzen \emph{reflektiert}, wenn folgendes gilt: Wann
immer~$F(f)$ eine schwache Äquivalenz ist, ist schon~$f$ eine schwache
Äquivalenz. Die Morphismen~$\eta$ und~$\varepsilon$ aus den letzten zwei
Bedingungen stammen von der Eins bzw. Koeins der Adjunktion.

\end{document}
