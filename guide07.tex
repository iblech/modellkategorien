\documentclass{uebblatt}

\begin{document}

\section*{Guide zu Übungsblatt 7}

Die Indexkategorie im Kolimes bei der Definition von Dichtheit in
\textbf{Aufgabe 1} sieht wie folgt aus. Die Objekte dieser Kategorie sind
\[ \text{Morphismen der Form~$K(m) \to c$ in~$\C$ mit~$m \in \M$ beliebig.} \]
Die Morphismen sind
\[ \Hom(K(m) \xra{f} c,\, K(n) \xra{g} c) \defeq
  \{ h : m \to n \,|\, g \circ K(h) = f \}. \]

Die vier Teilaufgaben von Aufgabe~1 sind voneinander unabhängig. Ich hoffe,
dass mindestens eine gefällt! Für \textbf{Teilaufgabe~b) von Aufgabe~1}
überlege dir gut, wie das fragliche Diagramm in diesem Fall aussieht. Skizze!
Welche Morphismen verlaufen in diesem Diagramm?

Für \textbf{Teilaufgabe~b) von Aufgabe~2} hole dir bei mir einen Tipp ab.
Minimaltipp vorab: Retrakte haben etwas mit idempotenten Morphismen zu tun.
Für Teilaufgabe~c) benötigt man möglicherweise die Voraussetzung, dass~$\C$
kovollständig ist (prüfe ich noch).

Verwende bei \textbf{Aufgabe 3a)} ohne Beweis, dass jeder nicht-diskrete Raum für keine
Kardinalzahl~$\kappa$ $\kappa$-kompakt ist. In Aufgabe~2 von Blatt~5 hatten wir
gesehen, dass nicht-diskrete Räume zumindest nicht $\aleph_0$-kompakt sind; der
Beweis dort überträgt sich auf die allgemeinere Behauptung.

Ich persönlich finde \textbf{Teilaufgabe b) von Aufgabe 3} spannender.
Wer den Grad an benötigter Technik reduzieren möchte, kann folgende Behauptung
beweisen.
\begin{quote}
Sei~$I$ eine Partialordnung.
Sei~$(V_i)_{i \in I}$ eine monotone Familie von abgeschlossenen Unterräumen
eines Banachraums~$V$ (\emph{monoton} heißt: $i \preceq i' \Rightarrow
V_i \subseteq V_{i'}$).
Sei ein Vektor aus dem topologischen Abschluss der Vereinigung der~$V_i$
gegeben:~$x \in \overline{\bigcup_i V_i}$.
\begin{itemize}
\item Setze von~$I$ nur voraus, dass~$I$ gerichtet ist.
Zeige anhand eines expliziten Gegenbeispiels, dass dann im Allgemeinen nicht
folgt, dass~$x \in V_j$ für ein~$j \in J$.
\item Setze nun an~$I$ voraus, dass~$I$ $\aleph_1$-gerichtet ist. (Das
bedeutet, dass jede Teilmenge~$J \subseteq I$ mit~$|J| < \aleph_1$ eine obere
Schranke in~$I$ besitzt.) Zeige, dass dann durchaus folgt, dass~$x \in V_j$ für
ein~$j \in J$.
\end{itemize}
\end{quote}
Die Zahl~$\aleph_1$ ist die nach~$\aleph_0$ nächstgrößere Kardinalzahl. Analog
wie für eine Menge~$X$ genau dann~$|X| < \aleph_0$ gilt, wenn~$X$ endlich ist,
gilt genau dann~$|X| < \aleph_1$, wenn~$X$ (endlich oder) abzählbar ist.

Wer die eigentliche Aufgabe bearbeite möchte, sollte sich zunächst überlegen,
wie filtrierte Kolimiten in der Kategorie der Banachräume und linearen
Kontraktionen berechnet werden. (Wie ist die Norm zu definieren? Ist die eigene
Vermutung über den Kolimes wieder vollständig?) Eine lineare Abbildung~$A$ zwischen
Banachräumen heißt genau dann \emph{Kontraktion}, wenn~$\|Ax\| \leq \|x\|$ für
alle~$x$ aus dem Quellraum.

Übrigens: Die Kategorie der Banachräume ist zwar nicht lokal
$\aleph_0$-präsentierbar, wohl aber lokal $\aleph_1$-präsentierbar. Das hat
sogar eine erstaunlich einfache Begründung (siehe Übung).

\end{document}
