\documentclass{uebblatt}

\begin{document}

\maketitle{13}{}

\end{document}

0) Ch_*(R) ist lokal präsentierbar.

1) Dold-Kan zwischen simplizialen R-Moduln und Ch_*(R). Transport der
Modellstruktur auf sMod(R). Beweis, daß sMod(R) simpliziale kombinatorische
Modellkategorie ist.

2) Quillen(bi-)funktoren, die durch Ringwechsel induziert werden.

3) Berechnung eines linksabgeleiteten Funktors, etwa \otimes

4) Berechnung des Homotopietypes von Map(X, Y) in einem interessanten
Beispiel

5) Diskussion anderer Modellstrukturen auf Ch_*(R)?
