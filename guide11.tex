\documentclass{uebblatt}

\begin{document}

\section*{Guide zu Übungsblatt 11}

In \textbf{Aufgabe 1} heißen zwei Diagramme~$X \leftarrow A \rightarrow Y$
und~$X' \leftarrow A' \rightarrow Y'$ genau dann zueinander schwach äquivalent,
wenn es ein kommutatives Diagramm der Form
\[ \xymatrix{
  X \ar[d] & A \ar[l]\ar[r] \ar[d] & Y \ar[d] \\
  X' & A' \ar[l]\ar[r] & Y'
} \]
gibt, in dem die vertikalen Morphismen schwache Äquivalenzen sind.

Verwende bei \textbf{Aufgabe 1a)} als Definition von "`schwache Äquivalenz"'
einfach "`Homotopieäquivalenz"'. (Oder "`schwache Homotopieäquivalenz"', der
Unterschied spielt für diese Teilaufgabe keine Rolle.) Ein Tipp: Denke an die
beiden Hälften eines Überraschungseis!

\end{document}
