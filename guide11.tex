\documentclass{uebblatt}

\begin{document}

\section*{Guide zu Übungsblatt 11}

In \textbf{Aufgabe 1} heißen zwei Diagramme~$X \leftarrow A \rightarrow Y$
und~$X' \leftarrow A' \rightarrow Y'$ genau dann zueinander schwach äquivalent,
wenn es ein kommutatives Diagramm der Form
\[ \xymatrix{
  X \ar[d] & A \ar[l]\ar[r] \ar[d] & Y \ar[d] \\
  X' & A' \ar[l]\ar[r] & Y'
} \]
gibt, in dem die vertikalen Morphismen schwache Äquivalenzen sind.

Verwende bei \textbf{Aufgabe 1a)} als Definition von "`schwache Äquivalenz"'
einfach "`Homotopieäquivalenz"'. (Oder "`schwache Homotopieäquivalenz"', der
Unterschied spielt für diese Teilaufgabe keine Rolle.) Ein Tipp: Denke an die
beiden Hälften eines Überraschungseis!

Stresst euch bei \textbf{Aufgabe 2a)} nicht zu sehr mit den mehreren Schichten
von Definitionen! Die nichtdegenerierten Simplizes von~$\Delta[1]$ sind zum
Beispiel (in kurzer praktischer Notation)~$[0]$,~$[1]$ und~$[0,1]$. Die Ordnung ist
so, dass~$[0]$ und~$[1]$ unvergleichbar sind und dass beide kleiner als~$[0,1]$
sind. Die nichtdegenerierten Simplizes von~$\Delta[2]$ sind
\[
  [0], \quad
  [1], \quad
  [2], \quad
  [0,1], \quad
  [0,2], \quad
  [1,2], \quad
  [0,1,2].
\]

Die Behauptung bei \textbf{Aufgabe 2c)} ist präzise ausformuliert folgende. Ist
irgendein Morphismus~$\Lambda^i[n] \to \mathrm{Ex}(X)$ gegeben, so existiert
ein Morphismus~$\Delta[n] \to \mathrm{Ex}^2(X)$, sodass das Diagramm
\[ \xymatrix{
  \Lambda^i[n] \ar[r] \ar@{^{(}->}[d] & \mathrm{Ex}(X) \ar[d]^{j_{\mathrm{Ex}(X)}} \\
  \Delta[n] \ar@{-->}[r] & \mathrm{Ex}^2(X)
} \]
kommutiert. Mit~"`$\mathrm{Ex}^2(X)$"' ist~$\mathrm{Ex}(\mathrm{Ex}(X))$ gemeint.

Verwende bei \textbf{Aufgabe 3} folgendes Lemma (ohne Beweis): Eine
Kan-Faserung ist genau dann trivial, wenn sie eine schwache Äquivalenz ist.
Wir wissen auch schon, dass anodyne Erweiterungen schwache Äquivalenzen sind.
Es ist also nur noch die Rückrichtung zu zeigen.

Beachte bei \textbf{Aufgabe 4}, dass~$\SS^1$ kein Kan-Komplex ist!

Verwende bei \textbf{Aufgabe 5} die \emph{freie Vervollständigung
unter~$\kappa$-kleinen Kolimiten}: Sei~$\I$ eine kleine Kategorie. Dann gibt es
eine Kategorie~$\overline{\I}$ zusammen mit einem volltreuen Funktor~$\I \to
\overline{\I}$ mit folgenden Eigenschaften:
\begin{enumerate}
\item[1.] $\overline{\I}$ enthält alle~$\kappa$-kleinen Kolimiten.
\item[2.] Ist~$F : \I \to \D$ irgendein Funktor in eine Kategorie,
die~$\kappa$-kleine Kolimiten enthält, so gibt es genau einen
Funktor~$\overline{F} : \overline{\I} \to \D$, der~$\kappa$-kleine Kolimiten
enthält und eingeschränkt auf~$\I$ mit~$F$ übereinstimmt.
\end{enumerate}
Die Kategorie~$\overline{\I}$ kann man übrigens explizit konstruieren als den
Abschluss des Bildes der Yoneda-Einbettung~$\I \to [\I^\op,\Set]$
unter~$\kappa$-kleinen Kolimiten.

\end{document}
